\documentclass[12pt]{article}

\usepackage{fullpage}
\usepackage[utf8]{inputenc}
\usepackage{amsthm,amsmath,amsfonts,amssymb}
\usepackage[english,russian]{babel}
\usepackage{hyperref}

\begin{document}

\section{Тестирующая система}

Автоматическая проверка решений домашних заданий студентов СПбАУ будет 
производится посредством проверяющей системы ``Яндекс.Контест''\footnote{Данный документ написан с использованием материалов московской Школы Анализа Данных компании ``Яндекс''.}. 
Система находится по адресу \url{http://contest.yandex.ru}.

Система предназначена для проведения соревнований по спортивному
программированию. Но мы будем ее использовать для проверки домашних
заданий. Чтобы найти домашнее задание, перейдите по ссылке
\url{http://contest.yandex.ru/contest/ContestList.html}.
Соревнование начинающееся со слов SPbAU Homework является
домашним заданием.

Для того, чтобы иметь право сдавать задачи, необходимо иметь зарегистрированный
логин на Яндексе, а также по этому логину быть зарегистрированным на домашнее 
задание. Регистрацию на домашнее задание производит семинарист. Не забудьте 
сообщить ему свой логин указанным им способом.

\textbf{Внимание!} Логин на яндексе должен содержать в качестве подстрок
ваши имя и фамилию.

\subsection{Интерфейс системы}
\label{interface}

Страничка контеста содержит несколько вкладок: задачи, посылки, положение участ-
ников и сообщения. Во вкладке задачи можно выбрать одну задачу из списка, скачать
ее условие или отправить на проверку.

В разделе Посылки вы можете просмотреть подробные данные об отправленных
вами в систему решениях, в частности, по ссылке View report можно, например, узнать
причины, по которым решение не компилируется на сервере, посмотреть исходный код
своего решения, а для засчитанной посылки  получить информацию о выполнении
на всех тестах.

Вкладка Положение участников показывает монитор со всеми участниками (сделавшими хоть 
одну посылку) и их посылками. На пересечении строки и столбца участник/задача стоит
\begin{itemize}
\item $-k$, если участник $k$ раз пробовал безуспешно сдать эту задачу.
\item $+k$, если задача сдана успешно с k + 1-й попытки.
\item Не стоит ничего, если попыток у участника по этой задаче не было. 
\end{itemize}

Score отвечает количеству сданных задач, Penalty в нашем случае не имеет значения. 
Участники ранжируются по Score. 

Во вкладке Сообщения можно задать нам вопрос по какой-либо задаче
или же посмотреть разосланные нами разъяснения по поводу каких-либо задач.


\subsection{Общая схема проверки решений}
\label{scheme}

Посылаемое вами решение по задаче проверяется автоматически на подготовленном 
наборе тестов (тест это файл со входными данными по задаче). Учтите,
в систему нужно отправлять единственный файл с решением. Ваша программа должна 
читать входные данные из стандартного потока ввода и писать в стандартный поток вывода. 
Входной формат, описанный в условии задач, гарантируется, а выходной
необходимо строго соблюдать (при этом лишние пробелы и переводы строк во многих
случаях не влияют на вердикт системы).
Если ваша программа успешно проходит проверку на всех тестах, то система выдает 
вердикт OK, и решение засчитывается. Не засчитанное решение получает один из
вердиктов, отличный от OK, а именно:
\begin{itemize}
\item \texttt{CE} (compilation error) ваша программа не компилируется на сервере.
\item \texttt{WA} (wrong answer) ваша программа выдает неверный ответ на каком-либо
тесте (номер теста системой указывается).
\item \texttt{TL} (time limit exceeded) ваша программа не успевает отработать за отведенное
ей время, указанное в условии. В таком случае наиболее вероятно то, что у вас
асимптотически неправильное решение. Возможно также, что в этом виноваты
баги или плохо написанный код (см. раздел \ref{advices}).
\item \texttt{ML} (memory limit exceeded) ваша программа использует слишком много памяти
(больше, чем указано в условии).
\item \texttt{RE} (runtime error) произошла ошибка выполнения программы. Причины мо-
гут быть, например, такие: чтение из файла и запись в файл, выход за границы
массива, деление на ноль.
\item \texttt{PE} (presentation error) устаревший вердикт, сообщающий об ошибке в формате
вывода. Как минимум он сообщает вам о том, что на соответствующем тесте ответ
неверный.
\end{itemize}

\textbf{Внимание!} Мы оставляем за собой право выставлять вердикт по решению
задачи вручную. Это может быть связано с дополнительными ограничениями 
на задачу. Просьба не перепосылать свое решение в систему, а в первую 
очередь выяснить, какие ограничения были нарушены, и исправить решение
должным образом.

\subsection{Советы по решению задач, их написанию и тестированию}
\label{advices}

Чтобы оценить время работы программы, можно считать, что в секунду она успеет 
выполнить порядка $10^8 - 10^9$  простейших операций. 
Тогда если вы придумали решение задачи, то исходя из его асимптотики несложно прикинуть, 
уложится оно по времени или нет: нужно оценить количество операций на максимальном по
времени работы тесте. К примеру,
если в ограничениях задачи $n \leq 10000$, и ваше решение имеет асимптотику
$O(n^2)$, то оно, вероятнее всего, уложится в секунду-две, равно как и решение 
с асимптотикой $O(n log n)$ при $n \leq 300000$.

Оценить примерное количество требуемой памяти обычно тоже несложно (например, 
исходя из размера примитивных типов данных). 

Теперь предположим, что у вас уже есть написанная программа. Перед тем, как 
посылать ее в систему, необходимо в обязательном порядке протестировать ее.
Полезными могут оказаться следующие тесты:

\begin{itemize}
\item Минимальные тесты, то есть тесты с минимально возможными входными дан-
ными. Их обычно легко ввести, и эти тесты могут отвечать за крайние случаи в
работе алгоритма.

\item Максимальные тесты, то есть тесты с максимально возможными входными дан-
ными. На этих тестах можно как проверить свою программу на ограничение по
времени и памяти (советуем выводить в debug output соответствующие данные),
так и на правильность результата.

\item Ручные тесты, то есть тесты, которые вы можете ввести и самостоятельно прове-
рить ответ для них.

\item Стресс-тест. Очень многие задачи, где требуется хорошая асимптотика, 
допускают тривиальные решения с плохой асимптотикой. В этом случае можно 
(и нужно) сверить выполнение своей быстрой программы с лобовым решением. 
При этом тесты для такой проверки можно генерировать каким угодно образом, 
например, случайным. Такой метод проверки правильности программы называется 
стресс-тестированием. 

\end{itemize}

В большинстве задач ограничение по времени установлено с запасом от нашего решения, 
тем не менее, может возникнуть ситуация, при которой ваше решение асимптотически 
правильно и все же не укладывается в TL. В таком случае, подумайте, написан ли ваш
код оптимально, не помогут ли ему неасимптотические оптимизации.

\end{document}
