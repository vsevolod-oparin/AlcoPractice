\documentclass[12pt]{article}

\usepackage{fullpage}
\usepackage[utf8]{inputenc}
\usepackage{amsthm,amsmath,amsfonts,amssymb}
\usepackage[english,russian]{babel}
\usepackage{hyperref}

\begin{document}

\section{Code review}

Автоматическая проверяющая система позволяет протестировать
любое решение как черный ящик: можно оценить корректность программы,
занимаемые ею время и память.

Незатронутыми остаются стиль написания кода, его читабельность/понимаемость.
Кроме того, посылаемые в систему решения иногда оказываются слишком
сложными и громоздкими, чем могли бы быть. Корректные решения могут различаться 
по размеру в десять раз.

В качестве дополнения к автоматической проверяющей системе мы проводим
процедуру \textit{Code review}. Оценивать код студентов будут сотрудники крупных
компаний, занимающиеся промышленным программированием.

\section{Процедура Code Review}

Каждому студенту назначается задача из проведенных
домашних заданий. Студенту надлежит оформить код
и отослать его на почтовый ящик \href{mailto:au.algorithmis@gmail.com}{au.algorithmis@gmail.com}.
Возможные языки программирования: Java или C/С++.

Тема письма должна быть строго следующего формата:
\begin{center}
\texttt{Code Review <Фамилия Имя> <Номер дз>.<Номер задачи>}
\end{center}
К письму следует приложить решение задачи, получающее \texttt{OK}
в проверяющей системе.

Общение с ревьюером кода подразумевает несколько итераций,
во время которых студент вносит исправления в свое решение.
Перед очередной посылкой на ревью необходимо
проверить работоспособность текущей версии программы 
(как минимум послать в проверяющую систему).

Программа считается зачтенной, если ревьюер сообщает,
что задача зачтена, и код получает \texttt{OK} в проверяющей системе.

\textbf{Внимание!} Пожалуйста, оставляйте переписку по конкретной
задаче в рамках одного треда. Если вам прислали замечание, и
вы внесли соответствующие правки, не пишите новое письмо.
Найдите старое с замечаниями и ответьте на него.

\section{Code style}

Стиль кода может варьироваться от одной команды разработчиков к другой.
Но всегда есть некоторые общие идеи, которые стоит знать будущему
разработчику.

Подробности вы узнаете во время Code Review. Общие представления
можно получить из любимого поисковика по ключевым слова 
\textit{style guide}, \textit{code}, <нужный язык программирования>.

\end{document}
