\section{\practicehead{Разделяй и властвуй. Основная теорема.}}

Рассказ про бинарный поиск.

\begin{enumerate}

  \item Определить асимптотику.
    \begin{itemize}
      \item $T(n) = 2 \cdot T(\lfloor \frac{n}{2} \rfloor) + n$.
      \item $T(n) = 2 \cdot T(\lfloor \frac{n}{2} \rfloor + 17) + n$.
      \item $T(n) = 2 \cdot T(\lfloor \sqrt{n} \rfloor) + \log n$.
      \item $T(n) = T(\lceil \frac{n}{2} \rceil) + 1$.
      \item $T(n) = T(a) + T(n - a) + n$ для произвольной константы $a$.
      \item $T(n) = T(\alpha \cdot n) + T((1 - \alpha) \cdot n) + n$ для произвольной
                константы $\alpha \in (0, 1)$.
      \item $T(n) = 4 \cdot T(\lfloor \frac{n}{2} \rfloor) + n^k$ для $k \in \{1, 2, 3\}$.
    \end{itemize}

  \item Провести объединение, пересечение, вычитание, юник над парой отсортированных
     массивов.

  \item Есть $k$ отсортированных массивов. В сумме массивы содержат $n$ элементов.
        Слить массивы за $O(n \log k)$.

  \item По двум отсортированным массивам длины $n$ и $m$ найти $k$-ый по величине
        элемент в объединении за $O(\log n + \log m)$.
  
  \item Дана посделовательность из $n$ натуральных различных чисел из отрезка $[0, n]$.
        В последовательности отсутствует ровно одно число. Разрешается сделать запрос 
        к одному биту одной ячейки массива. Требуется за линейное время найти
        недостающий элемент.

  \item Требуется придумать алгоритм с возможностью использования случайных битов для
        поиска медианы за линейное по матожиданию время. 

  \item Дана перестановка $p \in S_n$. Требуется найти число инверсий за $O(n \log n)$.

  \item Назовём множество точек на плоскости хорошим, если каждая пара удовлетворяет
        одному из трех критериев
        \begin{itemize}
          \item Пара точек лежит на одной горизонтальной прямой.
          \item Пара точек лежит на одной вертикальной прямой.
          \item В Bounding Box-е точек (включая границу) лежит хотя бы еще одна точка.
        \end{itemize}
        Дано множество из $n$ точек на плоскости. Требуется
        найти хорошее надмножество за $O(n \log n)$.

  \item Дано множество из $n$ точек на плоскости. Найти пару
        ближайших точек за $O(n \log n)$.
  
  \item Дано множество из $n$ векторов на плоскости. Разрешается
      координату любого вектора умножить на -1. Найти пару
      векторов, чья сумма минимальна.

%  \item Коммутатор.

  \item Дан граф $G = \langle V, E \rangle$. Разрешается удалить
      ровно одно ребро из графа. Какие ребра можно удалить, чтобы
      граф получился двудольным. Определить за $O(E \log E)$.

\end{enumerate}
