\section{\practicehead{Линейные алгоритмы.}}

\begin{enumerate}

  \item Реализовать стек с операциями \texttt{PUSH}, \texttt{POP}, \texttt{MAX}
        при условии, что каждая операция работает за константное время.

  \item Реализовать очередь с операциями \texttt{PUSH}, \texttt{POP}, \texttt{MAX}
        при условии, что каждая операция работает за константное время.

  \item Дана последовательность $a_1, a_2, \cdots, a_n \in \mathbb{Z}$.
        Найти $l, r$ ($1 \leq l \leq r \leq n$) такие, что сумма $\sum_{i = l}^r a_i$
        была бы максимальной. Задачу требуется решить за линейное от $n$ время.

  \item Дана последовательность $a_1, a_2, \cdots, a_n \in \mathbb{N}$ и $S \in \mathbb{N}$.
        Найти $l, r$ ($1 \leq l \leq r \leq n$) такие, что сумма $\sum_{i = l}^r a_i = S$.
        Задачу требуется решить за линейное от $n$ время.

  \item Дана последовательность $a_1, a_2, \cdots, a_n \in \mathbb{N}$.
    Найти $l, r$ ($1 \leq l \leq r \leq n$) такие, что значение $(r - l + 1) \min_{i \in [l, r]} a_i$
    было бы максимально. Задачу требуется решить за линейное от $n$ время.

  \item Дана последовательность объектов $a_1, a_2, \cdots, a_n$. Над объектами определена операция
        сравнения. Известно, что в последовательности есть элемент присутствующий не менее
        чем $\left\lfloor \frac{n}{2} \right\rfloor + 1$. Требуется найти элемент $a$ за линейное время
        и константу дополнительной памяти при условии, что последовательность задана а) массивом
        б) форвард-итератором.

  \item Дано число, представленное $n$ цифрами в десятичной записи без ведущих нулей.
      Из числа требуется вычеркнуть ровно $k$ цифр так, чтобы результат был бы максимальным.
      Задачу требуется решить за линейное от $n$ время.

  \item Игра. Коридор разбит на $n$ участков. Игрок находится перед первым участком. 
        Цель~--- пройти последний участок.  Игрок может двигаться только вперед, переходя
        с $i$-ого участка на $i+1$-ый.
        Изначально на участке $i$ находится $a_i$ плит. Если игрок уходит с участка $i$, значение
        $a_i$ уменьшается на единицу. Если игрок ступает на участок без плит, он умирает.
        На участке появляется $k$ плит. Игрок восстанавливается перед первым участком.
        По данным $n$, $k$, последовательностью $a_i$ определить какое число жизней и ходов
        потратит игрок, чтобы пройти коридор. Решить за линейное от $n$ время.

  \item Дан массив \texttt{a} длины $n$. К массиву разрешен доступ через две функции.
  \begin{itemize}
      \item \texttt{compare(i, j)} -- возращает $\texttt{true}$, если $\texttt{a[i]} > \texttt{a[j]}$. Иначе -- \texttt{false}.
      \item \texttt{swap(i, j)} -- меняет местами значения \texttt{a[i]} и $\texttt{a[j]}$.
  \end{itemize}
     Пусть время работы алгоритма $A$ ограниченно $c \cdot n$, где $c$~--- некоторая константа.
     Для произвольного алгоритма $A$ оценить 
      $$ 
          \textbf{Pr}_{p \leftarrow U(A_n)} \left[ A \text{ отсортирует массив } p \right].
      $$

\end{enumerate}
