\section{\practicehead{Задачи на поиск в глубину.}}

\begin{enumerate}

  \item (Easy) Дано дерево $T$ на $V$ вершинах. Двое играют в игру. На некоторой
  вершине стоит фишка. Каждый по очереди двигает эту фишку вдоль
  некоторого ребра. После того, как фишка покидает вершину, 
  вершина со всеми инцидентными ей ребрами из графа удаляется.
  Проигрывает тот, кто не может сделать ход. По заданному дереву
  и стартовой вершине определить кто выиграет при оптимальной 
  стратегии обоих игроков. Решить за $O(V)$.

  \item (Easy) Корневое дерево $T$ на $V$ вершинах задается массивом 
  из $V$ элементов. Все вершины пронумерованы. Для каждой вершины 
  в массиве указан его родитель.  Для корня $r$ значение в массиве
  равно -1. Требуется опредить как будет выглядеть новое представление 
  дерева, если корень $r$ сменить на корень $q$. Разрешается использовать
  $O(1)$ дополнительной памяти. Менять массив можно. Время $O(V)$.

  \item (Easy) Для заданного неориентированного графа $G$ посчитать
          минимальное число ребер, которые нужно добавить в граф, чтобы 
          он стал связным, за $O(V + E)$.

  \item (Easy) Проект задан набором задач. Задача под номером $i$ может
    быть выполнена ровно одним человеком за время $t_i$. Кроме того, у 
    каждой задачи есть зависимости. К задаче может приступить только 
    после того, как все задачи, от которой она зависит, оказались выполнены.
    Зависимости представлены ациклическим ориентированным графом. 
    Проект считается выполненным, если все его задачи выполнены.
    Требуется определить, за какое минимальное время можно управиться с проектом,
    если обладать неограниченным человеческим ресурсом. Время $O(V + E)$, где
    $V$~--- число задач, а $E$~--- число зависимостей.

  \item (Medium) Дан неориентированный граф $G$. Требуется разбить множество
      его вершин так, чтобы у каждой вершины был сосед в другой доле. Время
      $O(V + E)$.

  \item (Hard) Дан неориентированный граф $G$. Степень каждой вершины
      графа не превосходит трех. Требуется разбить граф на две доли так,
      чтобы у каждой вершине в ее же доле было не более одного соседа.
      Время $O(V + E)$.

  \item (Hard) Дан $d$ регулярный двудольный граф $G$ на $V$ вершинах. 
      Число $d$ является степенью двойки ($d = 2^k$). Требуется
      найти паросочетание в этом графе за $O(dV)$.

  \item (Hard) Для заданного ориентированного графа $G$ посчитать
        минимальное число ребер, которые нужно добавить в граф, чтобы
        он стал сильно связным. Время $O(V^3)$.

\end{enumerate}
