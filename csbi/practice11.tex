\section{\practicehead{Динамическое программирование.}}

\begin{enumerate}

  \item (Easy) Сеть закусочных О'Ducks планирует построить на трассе
  E95 несколько своих заведений. O'Ducks получило разрешение
  строить заведения на $m_1, m_2, \dots, m_n$ километрах.
  За каждое заведение известен профит, который получает O'Ducks -- 
  $p_1, p_2, \dots, p_n$. Министерство здравоохранения запретило
  строить два O'Ducks на расстоянии ближе чем $k$ километров.
  Помогите расположить заведения на возможных локациях, получив
  максимальный профит.

  \item (Easy) Дан словарь (набор слов) и текст длины $n$. С текстом случилось
  несчастье: пропали все пробелы и знаки пунктуации. Остались
  только буквы. Требуется используя словарь определить, можно
  ли разбить текст на отдельные слова и если можно, сколькими
  способами. Решить за $O(len \cdot n + size)$, где $len$ -- длина
  самого длинного слова в словаре, а $size$ -- сумма длин всех слов
  в словаре.

  \item (Medium) Дан группоид с таблицей умножения. В группоиде $g$ элементов.
  Дана произведение $n$ элементов. Вам разрешается ставить скобки
  где угодно. Требуется определить, какие элементы группоида можно
  получить в результате перемножения. Решить за $O(n^3 \cdot g^2)$.

  \item (Medium) Дан набор нечестных монеток с вероятностью выпадения орла
  $p_1, p_2, \dots, p_n$. Требуется посчитать вероятность выпадения
  ровно $k$ орлов за $O(n \cdot k)$. Обращение с числами считать 
  выполнимыми за $O(1)$.
  

  \item (Medium) Дано множество $S = \{s_1, \dots, s_n\}$. Требуется
  разбить $S$ на три множества $I$, $J$ и $K$ так, чтобы
  сумма во всех трех совпадала. Решить за $O(n \cdot (\sum_{s \in S} s)^2)$.

  \item (Medium) Дана последовательность $a_1, \dots, a_n$. Существуют
  два типа запросов.
  \begin{itemize}
    \item Развернуть подотрезок $[l, r)$ в обратном порядке.
    \item Вернуть $i$-ый элемент последовтельности.
  \end{itemize}
  Обработать $q$ запросов за $O(n \sqrt{n})$ (считать $q = O(n)$).

  \item (Medium) Федор купил Гикее выпуклую оболочку и теперь собирает ее 
  по инструкции. В комплект поставляется множество точек и инструкция.
  Инструкция состоит из $q$ пунктов двух типов.
  \begin{itemize}
    \item \texttt{add $p$} -- добавить точку $p$ в выпуклую оболочку.
    \item \texttt{check $p$} -- проверить лежит ли точка $p$ внутри выпуклой оболочки.
  \end{itemize}
  Результаты всех проверок Федор записывает на листочек. Помогите ему
  проверить, все ли верно сделал за $O(q \log q)$.
  
  \item (Medium) Роману подарили последовательность $a_1, \dots, a_n$.
  Роман обрадовался и решил посчитать число троек $(i, j, k)$
  таких, что $i < j < k$ и $a_i > a_j > a_k$. Решить за $O(n \log n)$.

\end{enumerate}
