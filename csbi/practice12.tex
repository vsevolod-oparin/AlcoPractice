\section{\practicehead{Структуры данных.}}

\begin{enumerate}

  \item (Easy) На плоскости расположено $n$ звезд.
  Каждая звезда имеет целочисленные координаты.
  Уровень звезды с координатами $(x, y)$ -- число
  звезд c координатами $(x', y')$ такими, что
  $x' \leq x$ и $y' \leq y$. Посчитать сколько звезд
  каждого уровня за $O(n \log n)$.

  \item (Easy) Есть память состоящая из $n$ блоков. Блоки могут
  быть занятыми и свободными.
  Менеджеру памяти в разные моменты времени приходит два типа
  запросов.
  \begin{itemize}
    \item \texttt{<Time> . Allocate} -- найти свободный блок с минимальным номером
    и аллоцировать его. В этом случае на протяжении времени $T$ блок считается занятым,
    потом свободным.
    \item \texttt{<Timie>. <TimeRequest>} -- запросить доступ к блоку памяти, аллокация 
    которого произошла в момент \texttt{<TimeRequest>}. Если блок уже свободен, менеджер
    сообщает \texttt{Fail}, иначе -- \texttt{Success}. В случае успеха время, которое 
    занят блок продлевается до $\texttt{<Timie>} + T$.
  \end{itemize}
  Требуется обслужить все запросы за $O(n \log n)$.

  \item (Medium) Дана последовательность $a_1, \dots, a_n$. Фиксирована 
  длина отрезка $l$. Заданы координаты $x_1, \dots, x_k$. Требуется
  для каждого отрезка $[x_i, x_i + l)$ найти максимальный элемент, 
  встречающийся в отрезке ровно один раз. Решить за $O(n \log n)$.

  \item (Medium) В плоскость по оси $Ox$ беспощадно вбили в ряд $n$ гвоздиков 
  (координаты $(1, 0)$, $(2, 0)$ $\dots$ $(n, 0)$). Напротив них с таким же успехов, 
  но с координатой $y = 1$ вбили еще $n$ гвоздиков. Продолжая вандализм, взяли $n$
  ниточек и связали пары гвоздиков так, что ни один гвоздик не повязан двумя ниточками,
  и каждая пара состоит из верхнего и нижнего гвоздика. Получилось паросочетание.
  По паросочетанию найти максимальное множество ниточек, каждая пара в котором пересекается.
  Решить за $O(n \log n)$.


  \item (Medium) Дана последовательность $a_1, \dots, a_n$. Для $t$ отрезков
  $[l_i, r_i)$ посчитаем $\sum_s K_s \cdot s$, где $s$ -- значение, присутствуещее
  в массиве, а $K_s$~-- количество раз, которое значение $s$ присутствует на отрезков.
  Решить за $O(n \sqrt n)$ или $O(n \log n)$.

  \item (Medium) В гардеробе есть $n$ крючков. Гардероб обслуживает $n$ человек.
  Может произойти три следующих события:
  \begin{itemize}
    \item \texttt{add $i$} -- повесить пальто персоны $i$ на крючок хитрым образом.
    \item \texttt{delete $i$} -- убрать пальто персоны $i$ с соответствующего крючка.
    \item \texttt{count $[l, r)$} -- посчитать число пальто на отрезке $[l, r)$. 
  \end{itemize}
  Хитрый образ выглядит так. Возьмем самый длинный отрезок пустых крючков. Если таких
  несколько, то возьмем самый правый. Там повесим пальто на центральный крючок.
  Если их два, вешаем на правый.
  Требуется обработать все запросы за $O(n \log n)$.

\end{enumerate}
