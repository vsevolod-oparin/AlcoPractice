\section{\practicehead{Задачи на поиск в глубину.}}

\begin{enumerate}

  \item (Easy)

	\item (Easy) Система оперирует плюшками. Каждая 
  плюшка имеет стоимость $2^{k_i}$. В каждый момент времени система
  может держать при себе ровно одну плюшку. Системе приходят либо 
  новые плюшки, либо запросы на плюшку стоимости $2^k$. Если система
  сейчас имеет в своем распоряжении плюшку стоимости $2^k$, то за 
  выполнения запроса система получает $2^k$ очков. Если приходит новая
  плюшка, системе надо сделать выбор: оставить старую или принять новую,
  а старую выкинуть. По истории запросов длины $n$ определить, какое 
  максимальное число очков могла набрать система.
	
  \item (Medium) Преподаватели делают заявки на занятия по времени вида $[b_i, e_i)$.
  В любой момент времени в одной аудитории может проходить максимум
  одно занятие. Распределить все заявки по аудиториям, используя 
  минимальное число аудиторий.

  \item () Работы на станках
  
  \item () Замена веса одного ребра

  \item () Заправки, бензобак. Минимальное число заправок

  \item () Игра с удалением ребер.
    
%  \item () Степени вершин.
\end{enumerate}
