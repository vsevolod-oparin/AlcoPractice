\section{\practicehead{Задачи на поиск в глубину.}}

\begin{enumerate}

	\item (Easy) Привести пример графа, на котором Деикстра не работает.
	
	\item (Easy) В стране $n$ аэропортов. Самолет может сделать перелет 
	  из аэропорта $i$ в аэропорт $j$, израсходовав $w_{ij}$ горючего.
	  При этом $w_{ij}$ может отличаться от $w_{ji}$, и $w_{ii} = 0$.
	  Требуется найти минимальный размер бака, позволяющий добраться
	  самолету из любого города в любой, возможно с дозаправками.
	  Решить за $O(n^2 \log W)$, где $W$~--- максимум по всем $w_{ij}$.

	\item (Easy) Дан ориентированный граф. Известно, что если убрать ориентации
	  у всех ребер, получится цикл. Стоимость разворота ребра $e$ состовляет
	  $p_e$. Требуется за минимальную стоимость сделать граф связным.

	\item (Medium) По заданному взвешанному графу $G$ найти цикл отрицательного
	  веса. Вывести его ребра. Время $O(VE)$.

	\item (Medium) Дана система на $n$ переменных, состоящая из $m$ неравенств
	  вида
	  $$
	  	x_i - x_j \leq \delta_{ij}.
	  $$
	  Требуется определить, выполнима ли система. Если да, привести хотя
	  бы одно решение.

    \item (Hard) Дан $d$~-регулярный граф $G$ на $n$ вершинах. 
	  $d$~--- степень двойки ($d = 2^k$). Найти паросочетание 
	  за $O(dN)$
	 
    \item (Hard) Дан неориентированный граф $G$. Посчитать
	  простые циклы длины $5$.

	\item (Hard) Дан неориентированный граф $G$. Двое играют
	  в игру. Изначально игроки $S$ и $T$ стоят в вершинах $s$ и $t$, соответственно. 
	  Затем, каждый игрок выбирает ребро $e$, инцидентное вершине, в которой
	  игрок находится, и за ход передвигается вдоль этого ребра.
	  Игрокам нельзя оказываться в один момент в одной вершине, но
	  можно в момент перехода оказаться на одном ребре. Требуется
	  привести игрока $T$ в вершину $s$, а игрока $S$~--- в вершину
	  $t$. В вершины игроки должны прийти одновременно. Стоять на месте
	  нельзя. Среди всех возможных вариантов найти решение с наименьшей
	  длиной пути. Решить за $O(V^2 + E)$.
	

\end{enumerate}
