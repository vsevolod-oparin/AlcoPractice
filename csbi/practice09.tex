\section{\practicehead{Динамическое программирование.}}

\begin{enumerate}

  \item (Easy) На билете есть $2n$ значный номер. Билет считается
  счастливым, если сумма первых $n$ цифр совпадает с суммой последних
  $n$ цифр. По заданому числу $n$ требуется найти число счастливых $n$
  значных билетов за $O(n^2)$. Считать, что стандартные арифметические
  операции над числами выполняютися за $O(1)$.

  \item (Easy) Задана система исчисления по основанию $K$. Требуется
  посчитать в такой системе число $n$ значных чисел без ведущих нулей 
  и нулей, повторяющихся два раза подряд.

  \item (Easy) Строка $s$ является палиндромом, если она читается
  одинаково как слева направо, так и справа налево. Требуется разбить
  строку $s$ длины $n$ на минимальное число палиндромов за $O(n^2)$.  
  
  \item (Medium) Дана строка $s$. Определим для нее сжатое представление.
  Пусть строка $s$ представима в виде конкатенации строки $s'$. 
  Например, $s = s's's'$. Тогда запишем $s$ как $(s')3$. Строка
  берется в скобки, а справа пишется, сколько раз требуется ее 
  повторить. Требуется найти для $s$ длины $n$ кратчайшее представление
  за $O(n^3)$.

  \item (Medium) Дан $n$-угольник. Каждая диагональ треугольника, соединяющая
  вершины $i$ и $j$ имеет вес $w_{ij}$. Вес триангуляциии многоугольника есть
  сумма весов диагоналей, которые в ней проведены. Найти триангуляцию с минимальным
  весом за $O(n^3)$.

  \item (Hard) Джип имеет размер бака 1 литр. На 1 км джим тратит 1 литр. Джип
  едет из пункта A в пункт B. Пусть расстояние между A и B будет $x$ километров. 
  Можно считать, что $x \leq 5$. В любой точке джип может оставить сколько-то бензина.
  В пункте A есть бесконечный источник бензина. Единственная доступная емкость -- бак.
  Найти, сколько бензина потребуется джипу, чтобы доехать до пункта B.
  Задача идейная, оценку не указываю.

  \item (Hard) Дано клетчатое поле $n \times m$. Соединять ребром можно только клетки,
  имеющие общую сторону. Треубется на поле построить гамильтонов цикл. Решить при
  условии, что $n \cdot m \leq 100$ и требуется совершить порядка $10^9$ операций.

\end{enumerate}
