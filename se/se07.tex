\section{\practicehead{Жадность. Хаффман}}

\begin{enumerate}

  \item Даны две строки $s_1$ и $s_2$. Какое минимальное 
  число раз надо повторить строку $s_1$, чтобы в ней встретилась
  строка $s_2$ в качестве подпоследовательности?

  \item Семен двигается из пункта A в пункт B на автомобиле.
  Расстояние между пунктами $D$ км. Автомобиля без подзаправки
  хватает на $d$ км. На расстояниях $x_1, x_2, \dots, x_n$ км 
  от пункта A находятся заправки. Найти минимальное число
  заправок, необходимых для того, чтобы автомобиль доехал
  до пункта B.

  \item Построить код Хаффмана, в котором частоты первых $n$
  символов совпадают с первыми $n$ числами Фибоначчи.

  \item Существует ли алгоритм, сжимающий хотя бы на один бит
  любой файл размером ровно в 1000 бит?

  \item Дан текст состоящий из $2^8$ различных символов. Каждый 
  символ присутствует в тексте. Частота любых двух символов 
  различается меньше чем в два раза. Доказать, что код Хаффмана
  не будет эффективней чем обычный 8-битный код (каждому символу
  соответсвует некоторое слово из 8 битов). Т.е. любой текст
  закодированный Хаффменом будет не короче, чем при помощи 8-битного
  кода.

\end{enumerate}


