\section{\practicehead{Динамика}}

\begin{enumerate}

  \item (Разобрано) Зафиксируем систему исчисления с основанием $K$. Требуется 
  посчитать количество $n$-значных чисел не содержащих двух подряд
  идущих нулей. Время $O(n^2)$.
  
  \item (Разобрано) Найти количетство замощений клетчатого поля $m \times n$ 
  доминошками $1 \times 2$ за $O(2^{2n} \cdot m)$.

  \item (Разобрано) Дан текст. Длина $i$-ого слова равна $l_i$. Требуется
  из текста сделать красивый абзац. В красивом абзаце каждая строка имеет
  длину $M$. Строку можно дополнять пробелами, между каждой парой слов в
  строке должен стоять хотя бы один пробел. 
  Гарантируется, что длина каждого слова $l_i \leq M$.\par
   В целях улучшения качества абзаца будем минимизировать следующую функцию. Для строки
  $j$ посчитаем число пробелов в ней, обозначим эту величину через $\text{gap}_j$.
  Требутся сделать абзац с наименьшим значением $\sum_j \text{gap}_j^3$.

  \item (Дома) Известно, что билет являтся счастливым, если сумма цифр первой
  половины его номера совпадает с суммой цифр второй половины. Найдите число
  счастливых билетов с $2n$-значными номерами за $O(n^3)$.

  \item (Дома) Есть клетчатое поле $n \times m$. Каждую клетку покрасим в
  черный или белый цвет. Покраска называется красивой, если она не содержит
  квадратов $2 \times 2$ одного цвета. Найдите число красивых покрасок за 
  $O(2^{2n}\cdot m)$.

  \item (Дома) Оптимизируйте поиск наибольшей возрастающей подпоследовательности
  от $O(n^2)$ до $O(n \log n)$, используя бинарный поиск.

  \item (Дома) Даны две последовательности длины $n$ и $m$. 
  Требуется найти общую подпоследовательность за $O(n \cdot m)$ времени и $O(n + m)$
  памяти. 
  

\end{enumerate}


