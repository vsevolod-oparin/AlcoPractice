\section{\practicehead{Жадность. Пути в графе.}}

\begin{enumerate}

  \item Постройте граф с отрицательными весами,
  на котором Деикстра выдавал бы неверный ответ.
  Граф не должен содержать отрицательных циклов.
  Где в доказательстве используется положительность
  весов ребер?

  \item Дана система неравенст на $n$ переменных.
  Каждое неравенство имеет вид $x_i - x_j \leq \delta_{ij}$.
  Всего неравенств $m$. Найти решение системы или сказать
  ,что его не существует, за $O(n \cdot m)$.

  \item Преподаватели сделали $n$ заявок на занятие. Каждое
  занятие начинается в момент $b_i$ и кончается в момент $e_i$
  (занимает интервал $[b_i, e_i)$). Два занятия в одной аудитории
  быть не могут. Распределите заявки по аудиториям так, чтобы
  общее число аудиторий было минимально. Решить за $O(n \log n)$.
  
  \item Дан взвешанный граф $G$. Дано минимальное остовное дерево
  на нем. У ребра $e$ поменяем вес на новый $w'$. По графу, остовному
  дереву, ребру и новому весу найдите новое остовное дерево за $O(V + E)$.

  \item Придумать реализацию алгоритма Деикстры за $O(V^2)$.

  \item Даны числа $d_1, d_2, \cdots d_n$. Построить граф без петель 
  и кратных ребер такой, что $i$-ая вершина имеет степень $d_i$.
  Если $m = \sum_{i = 1}^n d_i$, решить за $O(m^2)$.


\end{enumerate}


