\section{\practicehead{Линейные алгоритмы.}}

\begin{enumerate}

  \item Реализовать стек с операциями \texttt{PUSH}, \texttt{POP}, \texttt{MAX}
        при условии, что каждая операция работает за константное время.

  \item Реализовать очередь с операциями \texttt{PUSH}, \texttt{POP}, \texttt{MAX}
        при условии, что каждая операция работает за константное время.

  \item Дана последовательность $a_1, a_2, \cdots, a_n \in \mathbb{Z}$.
        Найти $l, r$ ($1 \leq l \leq r \leq n$) такие, что сумма $\sum_{i = l}^r a_i$
        была бы максимальной. Задачу требуется решить за линейное от $n$ время.

  \item Дана последовательность $a_1, a_2, \cdots, a_n \in \mathbb{N}$ и $S \in \mathbb{N}$.
        Найти $l, r$ ($1 \leq l \leq r \leq n$) такие, что сумма $\sum_{i = l}^r a_i = S$.
        Задачу требуется решить за линейное от $n$ время.

  \item Дана последовательность объектов $a_1, a_2, \cdots, a_n$. Над объектами определена операция
        сравнения. Известно, что в последовательности есть элемент присутствующий не менее
        чем $\left\lfloor \frac{n}{2} \right\rfloor + 1$. Требуется найти элемент $a$ за линейное время
        и константу дополнительной памяти при условии, что последовательность задана а) массивом
        б) форвард-итератором.

  \item Дано число, представленное $n$ цифрами в десятичной записи без ведущих нулей.
      Из числа требуется вычеркнуть ровно $k$ цифр так, чтобы результат был бы максимальным.
      Задачу требуется решить за линейное от $n$ время.

\end{enumerate}
