\section{\practicehead{Сортировки.}}

\begin{enumerate}
  \item Придумать алгоритмы вычисления и определить их асимптотику.
    \begin{itemize}
      \item $\lceil \sqrt{n} \rceil$.
      \item $\lceil \log{\log{n}} \rceil$.
      \item $\lceil \frac{a}{b} \rceil$.
    \end{itemize}

  \item По двум отсортированным массивам длины $n$ и $m$ найти $k$-ый по величине
        элемент в объединении за $O(\log n + \log m)$.     
        
  \item Придумать алгоритм разложения числа на простые множители, обосновать и определить асимптотику.
  
  \item Рассмотри устойчивую быструю сортировку. Пусть в качестве элемента разделителя всегда берётся средний элемент массива. (для массива чётной длины - выбирается элемент ближе к началу). Придумать алгоритм для построения трудного примера длины $n$.
	Пример труден, если глубина рекурсии максимальна относительно всех примеров длины $n$.
  
  \item Дано число из $n$ цифр. Требуется переставить цифры так, чтобы
	\begin{itemize}
		\item результат был без ведущих нулей;
		\item результат был минимальным.
	\end{itemize}
	Решить за $O(n)$.
	
  \item Последовательность $a_i$ длины $n$ записана слева направо.  
	Для каждого $a_i$ найдём самое правое $a_j$, меньшее $a_i$. Определим
	$r_i = j - i - 1$. Если такого $a_j$ нет, то $r_i = -1$. Требуется 
	посчитать по последовательности $a_i$ последовательность $r_i$ за $O(n \log n)$.
  
\end{enumerate}


